% interactcadsample.tex
% v1.03 - April 2017

\documentclass[]{interact}

\usepackage{epstopdf}% To incorporate .eps illustrations using PDFLaTeX, etc.
\usepackage{subfigure}% Support for small, `sub' figures and tables
%\usepackage[nolists,tablesfirst]{endfloat}% To `separate' figures and tables from text if required

\usepackage{natbib}% Citation support using natbib.sty
\bibpunct[, ]{(}{)}{;}{a}{}{,}% Citation support using natbib.sty
\renewcommand\bibfont{\fontsize{10}{12}\selectfont}% Bibliography support using natbib.sty

\theoremstyle{plain}% Theorem-like structures provided by amsthm.sty
\newtheorem{theorem}{Theorem}[section]
\newtheorem{lemma}[theorem]{Lemma}
\newtheorem{corollary}[theorem]{Corollary}
\newtheorem{proposition}[theorem]{Proposition}

\theoremstyle{definition}
\newtheorem{definition}[theorem]{Definition}
\newtheorem{example}[theorem]{Example}

\theoremstyle{remark}
\newtheorem{remark}{Remark}
\newtheorem{notation}{Notation}


% tightlist command for lists without linebreak
\providecommand{\tightlist}{%
  \setlength{\itemsep}{0pt}\setlength{\parskip}{0pt}}



\usepackage{hyperref}
\usepackage[utf8]{inputenc}
\def\tightlist{}


\begin{document}


\articletype{ARTICLE TEMPLATE}

\title{Size matters: Divergent morphology in elliptical bifaces from
sites in the American Southeast articulates with distinct local
reduction practices}


\author{\name{Robert Z. Selden, Jr.$^{a}$, John E. Dockall$^{b}$, David
H. Dye$^{c}$}
\affil{$^{a}$Heritage Research Center, Stephen F. Austin State
University; Department of Biology, Stephen F. Austin State University;
Texas Archeological Research Laboratory, The University of Texas at
Austin; and Cultural Heritage Department, Jean Monnet
University; $^{b}$Stantec, Inc.; $^{c}$Department of Earth Sciences, The
University of Memphis}
}

\thanks{CONTACT Robert Z. Selden,
Jr.. Email: \href{mailto:zselden@sfasu.edu}{\nolinkurl{zselden@sfasu.edu}}, John
E.
Dockall. Email: \href{mailto:john.dockall@stantec.com}{\nolinkurl{john.dockall@stantec.com}}, David
H.
Dye. Email: \href{mailto:daviddye@memphis.edu}{\nolinkurl{daviddye@memphis.edu}}}

\maketitle

\begin{abstract}
Elliptical bifaces are prevalent at Mississippian sites throughout the
American Southeast. Those from Millsap Cache and Jowell Farm (41AN13)
comprise two of the largest samples of this ill-understood stone tool
from the ancestral Caddo area. Elliptical bifaces from Millsap Cache
were produced using Kay County flint, while those from Jowell Farm were
manufactured using Edwards chert, providing for an empirical test of
morphological differences as a function of raw material. The sample of
elliptical bifaces was divided into two size classes; one conceptually
reflective of production (large), and the other with local reduction
practices (small). Size classes were used to assess whether
modifications by Caddo knappers may have yielded
similar---convergent---biface shape in the small class. Size classes
were also used to test the hypothesis that greater morphological
variation would be apparent in the small class due to idiosyncratic
responses related to local retouch practices. Results demonstrate that
elliptical biface shape does not differ by raw material, but size does.
This suggests that a shared, and consistent, mental template was
maintained independent of biface size. The subsequent analysis of
elliptical biface morphology by size class demonstrated that size does
not differ by raw material in the large class, but in the small class,
it does. This finding supports the argument that elliptical biface
morphology diverges through local reduction practices. As expected,
greater shape diversity occurs in the small class, where Jowell Farm
bifaces were found to be more morphologically diverse than those from
Millsap Cache. Distinct local reduction practices are advanced as the
driver of extant morphological differences found in elliptical bifaces.
\end{abstract}

\begin{keywords}
American Southeast; Caddo; NAGPRA; archaeology; ovoid biface; Jowell
knife; Jowell knives; lithics; oblique parallel flaking;
archaeoinformatics; museum studies; digital humanities; non-western art
history; geometric morphometrics; STEM; STEAM
\end{keywords}

\begin{quote}
\textit{This process of comparison, of recognising in one form a definite permutation or deformation of another, apart altogether from a precise and adequate understanding of the original 'type' or standard of comparison, lies within the immediate province of mathematics, and finds its solution in the elementary use of a certain method of the mathematician} \citep{RN7522}.
\end{quote}

\hypertarget{introduction}{%
\section{Introduction}\label{introduction}}

\begin{figure}\centering
\includegraphics[width=\linewidth]{figs/map.png}
\caption{Location of Millsap Cache, Jowell Farm, and other sites mentioned in the manuscript. Elliptical bifaces have been found in every major river basin in the ancestral Caddo area (white).}
\label{fig:map}
\end{figure}

\begin{figure}\centering
\includegraphics[width=\linewidth]{figs/elliptical.illustration.png}
\caption{Illustrated elliptical bifaces from a, an unknown location; b, Jowell Farm; and c, Belcher Mound highlighting oblique parallel flaking used in production. The specimen from Belcher Mound is a potential preform exhibiting oblique parallel removals consistent with elliptical bifaces from other sites in the region.}
\label{fig:illustrated}
\end{figure}

\hypertarget{methods}{%
\section{Methods}\label{methods}}

To assess the variable impacts that Caddo retouch practices may have had
on general morphology, the sample of elliptical bifaces was binned into
two size classes using centroid size. Due to potential morphological
differences associated with the use of two distinct raw materials (Kay
County flint and Edwards chert), the sample was subset by site prior to
calculating the mean centroid size for each. Bifaces larger than the
mean were then assigned to the large category, those smaller than the
mean were assigned to the small category, and the two datasets were then
joined in advance of analysis
(\href{https://seldenlab.github.io/elliptical.bifaces/define-size-classes.html}{Supplementary
Materials, Ch. 1}).

Elliptical bifaces from the Millsap Cache and Jowell Farm were used to
examine whether biface morphology remains stable or expresses
morphological variability by site/raw material
(\href{https://seldenlab.github.io/elliptical.bifaces/gm---siteraw-material.html}{Supplementary
Materials, Ch. 2}). Prior to landmarking, elliptical bifaces were
oriented with the most heavily retouched edge at top right (Figure
\ref{fig:elliptical}). Some bifaces include multiple areas of heavy
retouch at the top and bottom of the same lateral edge (Figure
\ref{fig:elliptical}:b, c, d, and e), while others (Figure
\ref{fig:elliptical}:f) include one retouched edge at top right and
another at bottom left.

\begin{figure}\centering
\includegraphics[width=\linewidth]{figs/ellipticalbifaces.png}
\caption{Selected elliptical bifaces from the Jowell Farm site highlighting the most heavily retouched area identified using a modified version of the flaking index developed and advanced by \citet{RN11099}, \citet{RN9242}, and \citet{RN11098}.}
\label{fig:elliptical}
\end{figure}

To identify which edge was most heavily retouched, we employed a
modified approach to the flaking index initially developed by
\citet{RN11099}, and later advanced and refined by \citet{RN9242}, then
\citet{RN11098}. The modified approach uses counts of flake scars from
each edge in the two most heavily worked areas, paired with a measure of
edge length inclusive of curvature, which was calculated using ImageJ
(Abramoff et al.~2004, Rasband 1997, Schneider et al.~2012). The number
of flake scars was then divided by the length of the worked edge, with
the most heavily worked edge identified by the greater value.

\hypertarget{geometric-morphometrics}{%
\subsection{Geometric morphometrics}\label{geometric-morphometrics}}

The same landmark/semilandmark configuration was used for both analyses,
and the landmarking protocol employs three landmarks; two horizontal
tangents (top/bottom), and the third placed at the furthest extent of
the retouched edge bearing the heaviest amount of retouch (see Figure
\ref{fig:elliptical}). Equidistant semilandmarks were placed between
each landmark, and all landmarks and semilandmarks were applied using R
4.2.1 (R Core Development Team, 2022) and the StereoMorph package (Olsen
and Westneat 2015).

Landmark data were aligned to a global coordinate system (Bookstein et
al.~1999, Gunz et al.~2005, Kendall 1981, 1984, Slice 2001), achieved
through generalised Procrustes superimposition (Bookstein 1986, Rohlf
and Slice 1990, Rohlf 1999) in R using the geomorph (Adams et al.~2017,
Adams and Otarola-Castillo 2013, Baken et al.~2021) and RRPP packages
(Adams and Collyer 2015, Collyer and Adams 2018). Procrustes
superimposition translates and rotates the coordinate data to allow for
comparisons among objects, while also scaling each biface using
unit-centroid size---the square root of the sum of squared distances
from each landmark to the specimen's centroid (Chapman 1990, Dryden and
Mardia 1998, Gower 1975, Rohlf and Slice 1990). The geomorph package
uses a partial Procrustes superimposition that projects the aligned
specimens into tangent space subsequent to alignment in preparation for
the use of multivariate methods that assume linear space (Dryden and
Mardia 1993, Kent and Mardia 2001, Rohlf 1999, Slice 2001).

\hypertarget{acknowledgements}{%
\section*{Acknowledgement(s)}\label{acknowledgements}}
\addcontentsline{toc}{section}{Acknowledgement(s)}

Our thanks to the Caddo Nation of Oklahoma, the Caddo Nation Tribal
Council, Tribal Chairman, and Tribal Historic Preservation Office for
permission and access to NAGPRA and previously repatriated collections.
Our gratitude is also extended to Marybeth Tomka and Lauren Bussiere at
the Texas Archeological Research Laboratory for their assistance with
access to the bifaces and associated records, to Sergio Ayala and the
Gault School of Archaeological Research for access to the UV light, and
to Scott Hammerstedt and Debra K. Green at the Oklahoma Archeological
Survey for their assistance with records requests. Thanks also to John
Harman for access to the DStretch plugin for ImageJ that was useful in
the analysis of flake scars, and to Harry J. Shafer, Hiram F. (Pete)
Gregory, Christian S. Hoggard, and David K. Thulman for their comments
and constructive criticisms on the ongoing analyses of Caddo biface
morphology, as well as Emma Sherratt, Kersten Bergstrom, Lauren Butaric,
Dean C. Adams, and Michael L. Collyer for their constructive criticisms,
general comments, and suggestions throughout the development of this
research program.

\hypertarget{funding}{%
\section*{Funding}\label{funding}}
\addcontentsline{toc}{section}{Funding}

Components of this analytical work flow were developed and funded by a
Preservation Technology and Training grant (P14AP00138) to RZS from the
National Center for Preservation Technology and Training (NCPTT), and
additional grants to RZS from the Caddo Nation of Oklahoma, National
Forests and Grasslands in Texas (15-PA-11081300-033) and the United
States Forest Service (20-PA-11081300-074). Funding to analyse the
bifaces from Millsap Cache and Jowell Farm was made possible by support
from the Heritage Research Center at Stephen F. Austin State University.
A Research Support Fund Grant from the Texas Archeological Society
provided the tool needed to illustrate the bifaces.

\hypertarget{data-management}{%
\section{Data management}\label{data-management}}

Reproducibility---the ability to recompute results---and
replicability---the chances other experimenters will achieve a
consistent result---are two foundational characteristics of successful
scientific research (Leek and Peng 2015). The analysis code associated
with this project can be accessed through the
\href{https://seldenlab.github.io/elliptical.bifaces/}{Supplementary
Materials}, is available through the GitHub repository
\url{https://github.com/seldenlab/elliptical.bifaces}, and digitally
curated on the Open Science Framework \href{https://osf.io/ph25w/}{DOI
10.17605/OSF.IO/PH25W}. The reproducible nature of this enterprise
provides a means for others to critically assess and evaluate the
various analytical components (Gray and Marwick 2019; Peng 2011; Gandrud
2014), which is a necessary requirement for the production of reliable
knowledge.

Reproducibility projects in psychology and cancer biology are impacting
current research practices across all domains. Examples of reproducible
research are becoming more abundant in archaeology (Marwick 2016;
Ivanovaite et al.~2020; Selden Jr., Dockall, and Shafer 2018; Selden
Jr., Dockall, and Dubied 2020; Selden Jr et al.~2021a; Selden 2022), and
the next generation of archaeologists are learning those tools and
methods needed to reproduce and/or replicate research results (Marwick
et al.~2019). Reproducible and replicable research work flows are often
employed at the highest levels of humanities-based inquiries to mitigate
concern or doubt regarding proper execution, and is of particular import
should the results have---explicitly or implicitly---a major impact on
scientific progress (Peels and Bouter 2018).

\bibliographystyle{tfcad}
\bibliography{interactcadsample.bib}


\input{"appendix.tex"}



\end{document}
